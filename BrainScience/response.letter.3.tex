\documentclass[journal,onecolumn,12pt]{IEEEtran} 

\usepackage{amsmath,amssymb,bm}
\usepackage{amsthm, amsfonts}	
\usepackage{bm,bbm}
\usepackage[normalem]{ulem}
\usepackage{color}
\usepackage{fancybox}
\usepackage{url,booktabs}
\usepackage[round]{natbib}

\usepackage{xr}



\title{Reply to Reviewer's Comments on\\
''EEG Waveform Analysis of P300 ERP with applications to Brain Computer Interfaces''}
\author{}

\begin{document}

\maketitle
\pagenumbering{roman}
\setcounter{page}{1}

We are grateful to the reviewer for giving us very valuable information about issues in our manuscript.

In the following, we discuss how we dealt with each raised issue. 

\vskip+1ex
\noindent \dotfill

\section*{\fbox{Reviewer \#2 Round 2 - Transcript:}}

The authors have addressed many of my concerns regarding this manuscript. There are however a few points that could benefit from clarification. 

1) While the authors have addressed the issue of having just a single null-EEG stream from a single subject, it appears to be the case still that a single P300 ERP template is being used. This would seem to either assuming that the P300 is generally invariant across trials (which is unlikely) or that the result of this assessment is restricted to the case in which the ERP response of interest does not vary between trials. This should at least be discussed as a limitation in the conclusion section of the manuscript. 

2) The write-up on the active modality section is currently slightly vague. For instance, the reader should not be required to consult Riccio et al. 2013 to obtain information on what "feedback" constitutes. Also, perhaps I am missing something, but I am slightly confused as to why pseudo-real datasets involving superimposition of a selected P300 waveform template are being constructed here. If this is a replication of Riccio, doesn't the original stream itself contains naturally occurring P300s which the authors could contrast the detection performance of their selected classification methods of choice on?

3) As indicated in my initial summary section, the value of this work to the clinical community remains unclear. The value of BCI in clinical work has a long and well-established history (e.g., Neuper et al. 2003). The reference cited in support of this in the response document (Chavarriaga et al., 2017) focuses on the challenges of long-term independent BCI use by intended end users which the current study does not seem to focus on.


\subsection*{\ovalbox{General Comments}}
The authors have addressed many of my concerns regarding this manuscript. There are however a few points that could benefit from clarification. 
\vskip+1ex
\noindent \dotfill
\vskip+1ex
%
\begin{quotation}
{\color{blue}

We thank again the Reviewer for all the detailed information provided on this fruitful Review.

}
\end{quotation}
%
\vskip+1ex
\noindent \dotfill
\vskip+1ex
1) While the authors have addressed the issue of having just a single null-EEG stream from a single subject, it appears to be the case still that a single P300 ERP template is being used. This would seem to either assuming that the P300 is generally invariant across trials (which is unlikely) or that the result of this assessment is restricted to the case in which the ERP response of interest does not vary between trials. This should at least be discussed as a limitation in the conclusion section of the manuscript. 
\vskip+1ex
\noindent \dotfill
\vskip+1ex
%
\begin{quotation}
{\color{blue}

We have modified the experiments to use all the available templates.  On the passive-modality, we are using now 70 averaged templates obtained from Subject Number 8 of the BNCI-Horizon dataset (Riccio et al).  They are randomly selected and they are superimposed on the EEG stream.  This subject was selected due to the fact that the waveforms of their averaged P300 responses have a very distinctive and prototypical shape.

For the active-modality, first all 70 templates are extracted from each subject by an offline averaging procedure.  These 70 templates are superimposed to a random non-hit preprocessed segment (see next issue).

}
\end{quotation}
%
\vskip+1ex
\noindent \dotfill
\vskip+1ex

2) The write-up on the active modality section is currently slightly vague. For instance, the reader should not be required to consult Riccio et al. 2013 to obtain information on what "feedback" constitutes. Also, perhaps I am missing something, but I am slightly confused as to why pseudo-real datasets involving superimposition of a selected P300 waveform template are being constructed here. If this is a replication of Riccio, doesn't the original stream itself contains naturally occurring P300s which the authors could contrast the detection performance of their selected classification methods of choice on?


\vskip+1ex
\noindent \dotfill
\vskip+1ex
%
\begin{quotation}
{\color{blue}

We have modified the Section 2.6 to clarify in detail the procedure.  We are using a pseudo-real dataset to have certainty of the existence of the P300 response, which is generating by a subject and we are not sure if the response is there or not.  This is the reason we followed a similar procedure performed by Varlaager in REF while trying to analyze different alternatives of alignment of P3 trials.  But instead of using a costumized completely artificial P300 waveform we used one obtained from a public dataset of a real person, and from a particular group which is targetted by BCI applications.


}
\end{quotation}
%
\vskip+1ex
\noindent \dotfill
\vskip+1ex

3) As indicated in my initial summary section, the value of this work to the clinical community remains unclear. The value of BCI in clinical work has a long and well-established history (e.g., Neuper et al. 2003). The reference cited in support of this in the response document (Chavarriaga et al., 2017) focuses on the challenges of long-term independent BCI use by intended end users which the current study does not seem to focus on.


\vskip+1ex
\noindent \dotfill
\vskip+1ex
%
\begin{quotation}
{\color{blue}
Thank you very much for this comment.  It pointed out something that we stated incorrectly and which required more thoughtful write-up. We refer to the approach suggested by Chavarriaga et al 2017 is the UCD, User-Centered Design, where they state that "close interaction between family caregivers and medical staff tremendously facilitates the porocess of transferring an AT to the end-user's home as not only can the needs and requirements of the end user can be taken into account but also the device can be adjusted to the needs and requiremens twho support the end user.  A multidisciplinary approach would allow for improvement of end-user training, for example, by taking into account expert opinion on how to create learning paradigms that promote success but also facilitate BCI use for the end-user." (page 4).

The message that we aimed to transmit, and that we hope this new version of the manuscript now incorporates, is that a waveform-based BCI system, which has an inherent intelligible property (REF), will foster a collaboration in a multidisciplinary environment and will ease the acceptance, and the usability and usefulness of a BCI device.   The reason being, for caregivers and medical staff, particularly those with the traditional knowledge of the clinical EEG which is based on waveforms, they will feel more natural the understanding of how the system is detecting signals.

We have modified the 2.6 section and the Conclusions.
}
\end{quotation}
%
\vskip+1ex
\noindent \dotfill
\vskip+1ex

\bibliographystyle{mdpi}
\bibliography{brainscience}

\end{document}
