\documentclass[journal,onecolumn,12pt]{IEEEtran} 

\usepackage{amsmath,amssymb,bm}
\usepackage{amsthm, amsfonts}	
\usepackage{bm,bbm}
\usepackage[normalem]{ulem}
\usepackage{color}
\usepackage{fancybox}
\usepackage{url,booktabs}
\usepackage[round]{natbib}

\usepackage{xr}



\title{Reply to Reviewer's Comments on\\
''EEG Waveform Analysis of P300 ERP with applications to Brain Computer Interfaces''}
\author{}

\begin{document}

\maketitle
\pagenumbering{roman}
\setcounter{page}{1}

We are grateful to the reviewer for giving us very valuable information about issues in our manuscript.

In the following, we discuss how we dealt with each raised issue. 

\vskip+1ex
\noindent \dotfill

\section*{\fbox{Reviewer \#2 Round 2 - Transcript:}}

The authors have addressed many of my concerns regarding this manuscript. There are however a few points that could benefit from clarification. 

1) While the authors have addressed the issue of having just a single null-EEG stream from a single subject, it appears to be the case still that a single P300 ERP template is being used. This would seem to either assuming that the P300 is generally invariant across trials (which is unlikely) or that the result of this assessment is restricted to the case in which the ERP response of interest does not vary between trials. This should at least be discussed as a limitation in the conclusion section of the manuscript. 

2) The write-up on the active modality section is currently slightly vague. For instance, the reader should not be required to consult Riccio et al. 2013 to obtain information on what "feedback" constitutes. Also, perhaps I am missing something, but I am slightly confused as to why pseudo-real datasets involving superimposition of a selected P300 waveform template are being constructed here. If this is a replication of Riccio, doesn't the original stream itself contains naturally occurring P300s which the authors could contrast the detection performance of their selected classification methods of choice on?

3) As indicated in my initial summary section, the value of this work to the clinical community remains unclear. The value of BCI in clinical work has a long and well-established history (e.g., Neuper et al. 2003). The reference cited in support of this in the response document (Chavarriaga et al., 2017) focuses on the challenges of long-term independent BCI use by intended end users which the current study does not seem to focus on.


\subsection*{\ovalbox{General Comments}}
The authors have addressed many of my concerns regarding this manuscript. There are however a few points that could benefit from clarification. 
\vskip+1ex
\noindent \dotfill
\vskip+1ex
%
\begin{quotation}
{\color{blue}

We thank again the Reviewer for all the detailed information provided on this fruitful Review.

}
\end{quotation}
%
\vskip+1ex
\noindent \dotfill
\vskip+1ex
1) While the authors have addressed the issue of having just a single null-EEG stream from a single subject, it appears to be the case still that a single P300 ERP template is being used. This would seem to either assuming that the P300 is generally invariant across trials (which is unlikely) or that the result of this assessment is restricted to the case in which the ERP response of interest does not vary between trials. This should at least be discussed as a limitation in the conclusion section of the manuscript. 
\vskip+1ex
\noindent \dotfill
\vskip+1ex
%
\begin{quotation}
{\color{blue}

We have modified the experiments to use all the available templates.  

On the passive-modality, we are using now 70 averaged templates obtained from the 35 trials produced by Subject Number 8 of the BNCI-Horizon dataset\citep{Riccio2013}.  They are randomly selected and they are superimposed on the EEG stream that we produced.  This subject was selected due to the fact that the waveforms of their averaged P300 responses have a very distinctive and prototypical shape.

Regarding the active-modality, first all 70 templates are extracted from each subject by an offline averaging procedure and from the segments marked as hits.  These 70 templates are superimposed to a random non-hit preprocessed segment (see next issue) to create new segments that are guaranteed to have the artificial P300 response.

We have modified completely Section 2.6 (Experimental Protocol) to provide details of these procedures.

}
\end{quotation}
%
\vskip+1ex
\noindent \dotfill
\vskip+1ex

2) The write-up on the active modality section is currently slightly vague. For instance, the reader should not be required to consult Riccio et al. 2013 to obtain information on what "feedback" constitutes. Also, perhaps I am missing something, but I am slightly confused as to why pseudo-real datasets involving superimposition of a selected P300 waveform template are being constructed here. If this is a replication of Riccio, doesn't the original stream itself contains naturally occurring P300s which the authors could contrast the detection performance of their selected classification methods of choice on?


\vskip+1ex
\noindent \dotfill
\vskip+1ex
%
\begin{quotation}
{\color{blue}

We are using a pseudo-real dataset to have certainty of the existence of the P300 response, which is generated by a subject and we are not sure if the response is there or not.  This is the reason we followed a similar procedure implemented by~\citep{Jaskowski2000} to analyze different alignment strategies of P3 trials.  Instead of using a completely artificial P300 waveform, we opted to use templates obtained from a public dataset of a real person and which, at the same time, have a very prototypical waveform.  Based on Reviewer suggestion we understood that the idea could be extended to use also the templates from the same subjects that performed a P300-Based BCI Speller experiment.

We have modified the Section 2.6 (Experimental Protocol) to clarify in detail the active modality.  We also modified the same section to emphasize the reason we opted for creating a pseudo-real dataset.  We hope it is now more clearly explained.

}
\end{quotation}
%
\vskip+1ex
\noindent \dotfill
\vskip+1ex

3) As indicated in my initial summary section, the value of this work to the clinical community remains unclear. The value of BCI in clinical work has a long and well-established history (e.g., Neuper et al. 2003). The reference cited in support of this in the response document (Chavarriaga et al., 2017) focuses on the challenges of long-term independent BCI use by intended end users which the current study does not seem to focus on.


\vskip+1ex
\noindent \dotfill
\vskip+1ex
%
\begin{quotation}
{\color{blue}
Thank you very much for this comment.  It pointed out something that we stated incorrectly and which required a more thoughtful write-up. 

We wanted to refer more specifically, to the recommendation suggested by \cite{Chavarriaga2017} to use the \textit{UCD, User-Centered Design}. In this work, authors state that "close interaction between family caregivers and medical staff tremendously facilitates the process of transferring an AT to the end-user's home as not only can the needs and requirements of the end user be taken into account but also the device can be adjusted to the needs and requirements of those who support the end user.  A multidisciplinary approach would allow for improvement of end-user training, for example, by taking into account expert opinion on how to create learning paradigms that promote success but also facilitate BCI use for the end-user." (page 4).

The message that we aimed to transmit, and that we hope this new version of the manuscript now incorporates, is that a waveform-based BCI system has an inherent intelligible property~\citep{j2018challenge}.  We mean the ability of the system to emphasize clearly and noticeable what are the factors that caused the system action, decision or classification, in this case the waveform that can be visually inspected. We think such a system will help to foster collaboration in a multidisciplinary environment and will ease the acceptance, the usability and usefulness of a BCI device.   The reason being, for caregivers and medical staff, particularly those with the traditional knowledge of the clinical EEG which is based on waveforms,  will feel more natural the understanding of how the system is detecting and analyzing signals.

We have modified the Conclusions section to emphasize our point.
}
\end{quotation}
%
\vskip+1ex
\noindent \dotfill
\vskip+1ex

\bibliographystyle{mdpi}
\bibliography{brainscience}

\end{document}
