\documentclass[journal,onecolumn,12pt]{IEEEtran} 

\usepackage{amsmath,amssymb,bm}
\usepackage{amsthm, amsfonts}	
\usepackage{bm,bbm}
\usepackage[normalem]{ulem}
\usepackage{color}
\usepackage{fancybox}
\usepackage{url,booktabs}
\usepackage[round]{natbib}

\usepackage{xr}



\title{Reply to Reviewer's Comments on\\
''EEG Waveform Analysis of P300 ERP with applications to Brain Computer Interfaces''}
\author{}

\begin{document}

\maketitle
\pagenumbering{roman}
\setcounter{page}{1}

We are grateful to the reviewer for pointing out relevant issues in our manuscript.

In the following, we discuss how we dealt with each raised issue. 

\vskip+1ex
\noindent \dotfill

\section*{\fbox{Reviewer \#2 Round 2 - Transcript:}}

The authors have addressed many of my concerns regarding this manuscript. There are however a few points that could benefit from clarification. 

1) While the authors have addressed the issue of having just a single null-EEG stream from a single subject, it appears to be the case still that a single P300 ERP template is being used. This would seem to either assuming that the P300 is generally invariant across trials (which is unlikely) or that the result of this assessment is restricted to the case in which the ERP response of interest does not vary between trials. This should at least be discussed as a limitation in the conclusion section of the manuscript. 

2) The write-up on the active modality section is currently slightly vague. For instance, the reader should not be required to consult Riccio et al. 2013 to obtain information on what "feedback" constitutes. Also, perhaps I am missing something, but I am slightly confused as to why pseudo-real datasets involving superimposition of a selected P300 waveform template are being constructed here. If this is a replication of Riccio, doesn't the original stream itself contains naturally occurring P300s which the authors could contrast the detection performance of their selected classification methods of choice on?

3) As indicated in my initial summary section, the value of this work to the clinical community remains unclear. The value of BCI in clinical work has a long and well-established history (e.g., Neuper et al. 2003). The reference cited in support of this in the response document (Chavarriaga et al., 2017) focuses on the challenges of long-term independent BCI use by intended end users which the current study does not seem to focus on.


\subsection*{\ovalbox{General Comments}}
The authors have addressed many of my concerns regarding this manuscript. There are however a few points that could benefit from clarification. 
\vskip+1ex
\noindent \dotfill
\vskip+1ex
%
\begin{quotation}
{\color{blue}

We thank the Reviewer for 

}
\end{quotation}
%
\vskip+1ex
\noindent \dotfill
\vskip+1ex
1) While the authors have addressed the issue of having just a single null-EEG stream from a single subject, it appears to be the case still that a single P300 ERP template is being used. This would seem to either assuming that the P300 is generally invariant across trials (which is unlikely) or that the result of this assessment is restricted to the case in which the ERP response of interest does not vary between trials. This should at least be discussed as a limitation in the conclusion section of the manuscript. 
\vskip+1ex
\noindent \dotfill
\vskip+1ex
%
\begin{quotation}
{\color{blue}
We apologize for this mistake.  We definitely selected an incorrect wording because the message that we aimed to convey was exactly the opposite.  We have modified the abstract to emphasize the message and the idea that we failed to transmit in our original manuscript.
}
\end{quotation}
%
\vskip+1ex
\noindent \dotfill
\vskip+1ex

2) The write-up on the active modality section is currently slightly vague. For instance, the reader should not be required to consult Riccio et al. 2013 to obtain information on what "feedback" constitutes. Also, perhaps I am missing something, but I am slightly confused as to why pseudo-real datasets involving superimposition of a selected P300 waveform template are being constructed here. If this is a replication of Riccio, doesn't the original stream itself contains naturally occurring P300s which the authors could contrast the detection performance of their selected classification methods of choice on?


\vskip+1ex
\noindent \dotfill
\vskip+1ex
%
\begin{quotation}
{\color{blue}
These important references were added to the manuscript.  Thank you for pointing them out.
}
\end{quotation}
%
\vskip+1ex
\noindent \dotfill
\vskip+1ex

3) As indicated in my initial summary section, the value of this work to the clinical community remains unclear. The value of BCI in clinical work has a long and well-established history (e.g., Neuper et al. 2003). The reference cited in support of this in the response document (Chavarriaga et al., 2017) focuses on the challenges of long-term independent BCI use by intended end users which the current study does not seem to focus on.


\vskip+1ex
\noindent \dotfill
\vskip+1ex


\bibliographystyle{mdpi}
\bibliography{brainscience}

\end{document}
