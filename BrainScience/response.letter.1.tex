\documentclass[journal,onecolumn,12pt]{IEEEtran} 

\usepackage{amsmath,amssymb,bm}
\usepackage{amsthm, amsfonts}	
\usepackage{bm,bbm}
\usepackage[normalem]{ulem}
\usepackage{color}
\usepackage{fancybox}
\usepackage{url,booktabs}
\usepackage[round]{natbib}

\usepackage{xr}



\title{Reply to Reviewer's Comments on\\
''EEG Waveform Analysis of P300 ERP with applications to Brain Computer Interfaces''}
\author{}

\begin{document}

\maketitle
\pagenumbering{roman}
\setcounter{page}{1}

We are grateful to the reviewer for pointing out relevant issues in our manuscript.

In the following, we discuss how we dealt with each raised issue. 

\vskip+1ex
\noindent \dotfill

\section*{\fbox{Reviewer \#1 Transcript:}}

The paper is an interesting research paper and presents an analysis of the P300 ERP. The authors provide a literature review of the current methods used to detect EEG patterns and conduct three experiments to identify the P300 ERP in EEG recordings. The paper is well-organized. The paper should incorporate the following comments before its acceptance for publication:

 

1. The major concern is that the experiments have been conducted with only one participant. The authors should collect more data in this study. In other studies cited in the paper by authors,  two and 16 participants were recruited.

 

2. It is strongly recommended that the authors should re-check some paragraphs of the manuscript, correct any grammatical and syntactic errors and improve the scientific language. In some cases, it was hard to follow the meaning. (e.g. Page 2, Paragraph 4, Line 51-54, Page 11, Paragraph 3, Line 343-345)

 

3. In the Introduction section the authors refer to BCI applications; however, no scientific studies are reported. Also, the literature section should be revised with recent publication. Below are some scientific studies that is suggested to be included for BCI applications: 

 

a. Cruz, A., Pires, G., \& Nunes, U. J. (2018). Double ErrP Detection for Automatic Error Correction in an ERP-Based BCI Speller. IEEE Transactions on Neural Systems and Rehabilitation Engineering, 26(1), 26-36.

 

b. Tzimourta, K.D., Tsoulos, I., Bilero, T., Tzallas, A., Tsipouras, M., Giannakeas, N. (2018). Direct Assessment of Alcohol Consumption in Mental State Using Brain Computer Interfaces and Grammatical Evolution. Inventions, 3(3), 51.

 

c. Kevric, J., \& Subasi, A. (2016). The impact of Mspca signal de-noising in real-time wireless brain computer interface system. Southeast Europe Journal of Soft Computing, 4(1).

 

and for sleep studies: 

 

d. Boostani R, Karimzadeh F, Nami M (2017) A comparative review on sleep stage classification methods in patients and healthy individuals. Computer methods and programs in biomedicine 140: 77-91.

 

e. Dimitriadis, S. I., Salis, C., \& Linden, D. (2018). A novel, fast and efficient single-sensor automatic sleep-stage classification based on complementary cross-frequency coupling estimates. Clinical Neurophysiology, 129(4), 815-828.

 

4. In the literature section, Table I briefly presents the proposed methods followed by other authors; however, there is no text describing the table. The authors should include a paragraph describing each method and the recent findings concisely, in order to present better the literature review. 

 

5. In Figure 2 please insert the y-axis title to clearly present ERP template.

 

6. The Results section is not well-presented. The authors are advised to include a paragraph before Table 2, Table 3 and Fig. 6-8 presenting the obtained results (best/worst channel performance and algorithm performance for each experiment) and then discuss the results in the Discussion section. Please revise excessive descriptions in the captions of Table 2, Table 3 and Fig. 6-8.

 

7. “the obtained character identification rate is above theoretical chance level, and for some algorithms close to the usable threshold of 70\%” (Page 19, Paragraph 2, Line 465-466) What is the obtained character identification rate in other scientific studies? Is the obtained result (70\%) of this study comparable to others?

 

8. In the Discussion section a table presenting the statistical analysis results would present better the results and would facilitate the reader.


\subsection*{\ovalbox{General Comments}}
The paper is an interesting research paper and presents an analysis of the P300 ERP. The authors provide a literature review of the current methods used to detect EEG patterns and conduct three experiments to identify the P300 ERP in EEG recordings. The paper is well-organized. The paper should incorporate the following comments before its acceptance for publication:
\vskip+1ex
\noindent \dotfill
\vskip+1ex
%
\begin{quotation}
{\color{blue}
Thank you for your encouraging and thoughtful comments. We have modified several points according to your recommendations. We hope the article is now better suited for publication. 
}
\end{quotation}
%
\vskip+1ex
\noindent \dotfill
\vskip+1ex

1. The major concern is that the experiments have been conducted with only one participant. The authors should collect more data in this study. In other studies cited in the paper by authors,  two and 16 participants were recruited.

\vskip+1ex
\noindent \dotfill
\vskip+1ex
%
\begin{quotation}
{\color{blue}
We have now included more participants to this experiment and it was divided in two modalities.  The first, which is now called passive-modality, includes now 4 subjects.  The second, an active-modality, now includes 4 more participants who were engaged in a copy-spelling task on a P300-Based BCI Speller.   For these subjects, the pseudo-real dataset is constructed by superimposed a P300 template obtained for each subject into segments marked as hit.  The baseline for each segment is constructed from a random no-hit segment for the same subject.  
}
\end{quotation}
%
\vskip+1ex
\noindent \dotfill
\vskip+1ex

2. It is strongly recommended that the authors should re-check some paragraphs of the manuscript, correct any grammatical and syntactic errors and improve the scientific language. In some cases, it was hard to follow the meaning. (e.g. Page 2, Paragraph 4, Line 51-54, Page 11, Paragraph 3, Line 343-345)


\vskip+1ex
\noindent \dotfill
\vskip+1ex
%
\begin{quotation}
{\color{blue}
Thank you very much for your comment.  We have modified the specific paragraphs that the Reviewer kindly offered as example and we have revised the entire manuscript modifying different parts, simplifying sections.
}
\end{quotation}
%
\vskip+1ex
\noindent \dotfill
\vskip+1ex

3. In the Introduction section the authors refer to BCI applications; however, no scientific studies are reported. Also, the literature section should be revised with recent publication. Below are some scientific studies that is suggested to be included for BCI applications: 

\vskip+1ex
\noindent \dotfill
\vskip+1ex
%
\begin{quotation}
{\color{blue}

Absolutely thank you very much for providing these very important references.  We added all of them.

}
\end{quotation}
%
\vskip+1ex
\noindent \dotfill
\vskip+1ex

4. In the literature section, Table I briefly presents the proposed methods followed by other authors; however, there is no text describing the table. The authors should include a paragraph describing each method and the recent findings concisely, in order to present better the literature review. 


\vskip+1ex
\noindent \dotfill
\vskip+1ex
%
\begin{quotation}
{\color{blue}

We have included a new paragraph describing the depictions which are shown in the table and in which areas are they used for.  

}
\end{quotation}
%
\vskip+1ex
\noindent \dotfill
\vskip+1ex

5. In Figure 2 please insert the y-axis title to clearly present ERP template.

\vskip+1ex
\noindent \dotfill
\vskip+1ex
%
\begin{quotation}
{\color{blue}
Our applogize for this mistake.  We have corrected it and revised all the other figures.
}
\end{quotation}
%
\vskip+1ex
\noindent \dotfill
\vskip+1ex

6. The Results section is not well-presented. The authors are advised to include a paragraph before Table 2, Table 3 and Fig. 6-8 presenting the obtained results (best/worst channel performance and algorithm performance for each experiment) and then discuss the results in the Discussion section. Please revise excessive descriptions in the captions of Table 2, Table 3 and Fig. 6-8.

\vskip+1ex
\noindent \dotfill
\vskip+1ex
%
\begin{quotation}
{\color{blue}
We have included the best/worst channel obtained for each algorithm, subject and experiment. We hope this enhancement contributes greatly to the understanding of the obtained values.  We have also modified the captions of the tables and figures, thank you very much for raising this issue.

 }
\end{quotation}
%
\vskip+1ex
\noindent \dotfill
\vskip+1ex

7. “the obtained character identification rate is above theoretical chance level, and for some algorithms close to the usable threshold of 70\%” (Page 19, Paragraph 2, Line 465-466) What is the obtained character identification rate in other scientific studies? Is the obtained result (70\%) of this study comparable to others?

\vskip+1ex
\noindent \dotfill
\vskip+1ex
%
\begin{quotation}
{\color{blue}
The reported binary classification accuracy for this BCI Competition was around 100\%. However in this case, as we are emphasizing the waveforms of the signals, the classification uses only single-channel activity, which is less information.  Regardless of this, the obtained results for some of the algorithms can be used to implement usable P300-Based Speller applications due to the utilization of word predictive schemes.
}
\end{quotation}
%
\vskip+1ex
\noindent \dotfill
\vskip+1ex

8. In the Discussion section a table presenting the statistical analysis results would present better the results and would facilitate the reader.

\vskip+1ex
\noindent \dotfill
\vskip+1ex
%
\begin{quotation}
{\color{blue}
We have now included more subjects and decided to remove the statistical analysis sections which was prepared for only one subject.
}
\end{quotation}
%

\bibliographystyle{mdpi}
\bibliography{article}

\end{document}